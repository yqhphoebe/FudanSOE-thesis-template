% !TeX program = xelatex
% !TeX encoding = UTF-8

%%重要!!!:github直接用overleaf打开的话,记得将compiler改为XeLaTex,直接用默认编译器可能无法编译

%% 加载自定义类 fdse.cls
\documentclass{fdse}

%% 导入参考文献数据
\addbibresource{reference.bib}

%%---------------------------------------------------------------------
%% 论文信息录入
%%---------------------------------------------------------------------
\studentID{000000}                  % 学号 
\degreecategory{博{ }士{ }学{ }位{ }论{ }文}            % 类别: 硕士学位论文 / 博士学位论文
\degreetype{(学术学位)}                  % 类型: (学术学位) / (专业学位) 
\thesistitle{中文题目} % 中文题目 (<30字) 
\englishtitle{English Title} % 英文题目
\major{专业名}                          % 学科专业 
\authorname{学生名}                     % 姓名
\supervisor{某 \quad 教授}         % 指导教师
\completedate{2026年5月20日}             % 完成日期

%%---------------------------------------------------------------------
%% 正文内容
%%---------------------------------------------------------------------
\begin{document}

% 1. 生成封面 
\makecover

% 2. 独创性声明 (这里仅为排版,正式提交通常需替换为扫描件) 
\thispagestyle{empty}
\begin{center}
    \zihao{3}\bfseries 复旦大学 \\ 学位论文独创性声明
\end{center}
\vspace{1em}
\noindent 本人郑重声明:所呈交的学位论文,是本人在导师的指导下,独立进行研究工作所取得的成果...
\clearpage

% 3. 前置部分 (摘要、目录) - 页码为罗马数字 I, II... 
\frontmatterstyle

% 指导小组 (可选) 
\chapter*{指导小组成员名单}
\begin{center}
    *** 教授 复旦大学经济学院 \\
    *** 教授 复旦大学经济学院
\end{center}

% 中文摘要 
\begin{cnabstract}{关键字1;关键字2;关键字3} % 填入关键词
本文研究了...
\end{cnabstract}

% 英文摘要 
% 参数1: Keywords, 参数2: JEL Classification
\begin{enabstract}{Keyword1; Keyword2; Keyword3}{G1; G1}
This paper investigates......
\end{enabstract}

% 目录 
\tableofcontents

% 4. 正文部分 - 页码为阿拉伯数字 1, 2... 
\mainmatter

\chapter{引言}
\section{研究背景}
这是正文段落,字体为小四号宋体,行距固定20磅 。

参考文献引用示例:
\begin{itemize}
    \item 中文单一作者:\parencite{xu2010}
    \item 英文单一作者:\parencite{barro1990}
    \item 英文多人作者:\parencite{acemoglu2001a}
\end{itemize}

\chapter{理论模型}
数学公式必须居中,编号右对齐,按章排序 :
\begin{equation}
    Y = \alpha + \beta X + \epsilon
    \label{eq:model}
\end{equation}

\chapter{实证结果}

% 表格示例
\begin{table}[t!]
    \centering
    \caption{模型估计结果}
    \zihao{5}
    \begin{tabular}{lccc}
        \toprule
         & OLS & FE & RE \\
        \midrule
        Log(labor) & 0.450*** & 0.096* & 0.059 \\
        Constant   & 0.489*** & 4.86*** & 5.66*** \\
        \bottomrule
    \end{tabular}
    \par\vspace{3pt}
    \raggedright\zihao{-5} 注:*** 表示 1\% 显著水平。
\end{table}

% 图形示例 
\begin{figure}[t!]
    \centering
    \fbox{\parbox{0.6\textwidth}{\centering \vspace{2cm} 图形占位 \vspace{2cm}}}
    \par\vspace{3pt}
    \raggedright\zihao{-5} 注:数据来源...
    \centering
    \caption{消费占GDP比例}
\end{figure}

\chapter{结论}
本文结论如下...

% 5. 参考文献 
\chapter{参考文献}
\printbibliography[check=chinese, heading=none] 
\printbibliography[check=english, heading=none] 

% 6. 附录 
\appendix
\chapter{数据说明}
附录内容...

% 7. 后记
\backmatter
\chapter{后记与致谢}
感谢...

\end{document}
